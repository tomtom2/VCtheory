\documentclass[a4paper]{report}

\usepackage[utf8]{inputenc}
\usepackage[francais]{babel}
\usepackage[T1]{fontenc}
\usepackage{amsmath}
\usepackage{amsfonts}
\usepackage{amssymb}
\usepackage{placeins}
\usepackage{listings}
\usepackage{color}
\usepackage{textcomp}

%----- Package français
\usepackage[utf8]{inputenc} %reconnaissance des accents
\usepackage[francais]{babel} %document en français
\usepackage[T1]{fontenc} %codage des fonts TeX ?



%----- math
\usepackage{amsmath}
\usepackage{dsfont}
\usepackage{calrsfs}

%----- images
\usepackage{graphicx}

\title{VC theory}
%\author{Thomas BRIEN}
%\date{28 janvier 2013}

\begin{document}
\maketitle
\tableofcontents

\renewcommand{\thesection}{\arabic{section}}

\newtheorem{theorem}{Theorem}
\newtheorem{lemma}{Lemme}

\renewcommand{\thetheorem}{\empty{}}
\renewcommand{\thelemma}{\empty{}} 

\newenvironment{proof}[1][Proof]{\begin{trivlist}
\item[\hskip \labelsep {\bfseries #1}]}{\end{trivlist}}
\newenvironment{definition}[1][Definition]{\begin{trivlist}
\item[\hskip \labelsep {\bfseries #1}]}{\end{trivlist}}
\newenvironment{example}[1][Example]{\begin{trivlist}
\item[\hskip \labelsep {\bfseries #1}]}{\end{trivlist}}
\newenvironment{remark}[1][Rq:]{\begin{trivlist}
\item[\hskip \labelsep {\bfseries #1}]}{\end{trivlist}}


\chapter*{Introduction}
\addcontentsline{toc}{chapter}{Introduction} 
Les différents concepts abordés sur la VC, ou Vapnik–Chervonenkis théorie: (fin 1970)
\begin{itemize}
\item Dimension VC
\item Pulvérisation
\item Échantillon Fantôme
\item Lemme de Sauer
\end{itemize}
On veut, avec une probabilité de $1 - \delta$:\\
$ \forall f \in F $ R(f)$ \leq $\^R (f)$ + \epsilon(\delta, F, n) $\\
\begin{tabbing}
avec : \= R(f) = $\mathds{P}_{X,Y \sim D}$(f(x)$\neq$y)\\
\> \^R(f) = $\frac{1}{n}\displaystyle { \sum_{i=1}^{n}}( \mathds{1}_{f(x_i)\neq y_i} $)\\
\end{tabbing}

\chapter*{Dimension VC}
\addcontentsline{toc}{chapter}{Dimension VC}

\section{Rappels et quelques exemples}
\subsection*{Rappels}
Lorsque |F| < $ +\infty $: avec proba 1-$ \delta $\\
$ \forall $ f $ \in $ F R(f)$ \leq $\^R(f) + $\sqrt{\frac{2}{n}*\log\frac{|F|}{\delta}}$\\
\newline
\underline{Preuve}:\\
\begin{itemize}
\item[$\bullet$] pour f fixé: $\mathds{P}($R(f)-\^R(f)$\geq \epsilon)\leq \exp ^{-2n\epsilon ^2}$
\item[$\bullet$] Borne de l'union: $\mathds{P}(\exists f\in F$ R(f)-\^R(f)$\geq \epsilon)\leq $ |F|$\exp ^{-2n\epsilon ^2}$\\
\end{itemize}

\begin{tiny}
Rappel du problème:\\
S = \{($X_i$, $Y_i)\}_{i=1}^n$ IID $ \sim $ D\\
\^R(f) = $\frac{1}{n}\displaystyle { \sum_{i=1}^{n}}( \mathds{1}_{f(x_i)\neq y_i} $)\\
R(f) = $\mathds{E}(\mathds{1}_{ f(x)\neq y })$ = $\mathds{P}$(f(x)$\neq$y)\\
\end{tiny}


\subsection*{Exemples sur deux classes de fonctions}
\begin{itemize}
\item[$\bullet$] $F_1 = \{f:\ f(x)=\mathds{1}_{x\geq t},\ t\in \mathds{R}\}
\ \cup \ \{f:\ f(x)=\mathds{1}_{x\leq t},\ t\in \mathds{R}\}$\\
\end{itemize}

\begin{itemize}
\item[$\bullet$] $F_2 = \{f:\ f(x)=\mathds{1}_{x\in A},\ A=[a,b]\times[c,d],\ a,b,c,d\ \in \ \mathds{R}\}$\\
\end{itemize}

\section{Théorème principale}
\begin{itemize}
\item[$\bullet$] F est une classe de fonctions $ \subseteq \mathcal{Y}^\mathcal{X}$\\
\item[$\bullet$] $\mathcal{N}_F (x_1..x_n)$ = $\{[f(x_1)..f(x_n)]\in \{0,i\}^n \ ,f\in F\}$\\
\end{itemize}

\begin{definition}
Coefficient de pulverisation $\mathcal{S}$(F, n)\\
$\mathcal{S}$(F,n) = $\underset{x_1..x_n}{max}|\mathcal{N}_F (x_1..x_n)|$\\
\end{definition}

\begin{remark}
$\mathcal{S}$(F,n) $ \leq 2^n$ mais bien souvent $\mathcal{S}$(F,n) $ \lll 2^n $\\
\end{remark}

\begin{definition}
Dimension de Vapnik–Chervonenkis\\
le plus grand k tq. $\mathcal{S}$(F,k) = $2^k$\\
\end{definition}

La VC dimension d'une classe de fonctions F, qu'on  notera VCdim(F), mesure la "richesse" de cette classe.\\
Si on revient sur nos exemples précédents:
\begin{itemize}
\item[]VCdim($F_1$) = 2
\item[]VCdim($F_1$) = 4\\
\end{itemize}

\begin{theorem}
VC (1979):\\
$\mathds{P}(sup|$\^R(f)-R(f)$| > \epsilon)\ \leq \ 8\mathcal{S}(F,n) \exp ^{-\frac{n\epsilon^2}{32}}$\\
$\mathds{P}(\exists f \in F:\ |$\^R(f)-R(f)$| > \epsilon)\ \leq \ 8\mathcal{S}(F,n)\exp ^{-\frac{n\epsilon^2}{32}}$\\
\end{theorem}

\begin{lemma}
Sauer-Shelah:\\
$\mathcal{S}$(F,n) $ \leq (n+1)^{VCdim(F)}$
\end{lemma}

\section{Preuve du théorème principale}

\end{document}