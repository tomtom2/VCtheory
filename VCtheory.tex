\documentclass[a4paper]{report}

\usepackage[utf8]{inputenc}
\usepackage[francais]{babel}
\usepackage[T1]{fontenc}
\usepackage{amsmath}
\usepackage{amsfonts}
\usepackage{amssymb}
\usepackage{placeins}
\usepackage{listings}
\usepackage{color}
\usepackage{textcomp}

%----- Package français
\usepackage[utf8]{inputenc} %reconnaissance des accents
\usepackage[francais]{babel} %document en français
\usepackage[T1]{fontenc} %codage des fonts TeX ?



%----- math
\usepackage{amsmath}
\usepackage{dsfont}
\usepackage{calrsfs}

%----- images
\usepackage{graphicx}

\title{VC theory}
%\author{Thomas BRIEN}
%\date{28 janvier 2013}

\begin{document}
%\maketitle
\tableofcontents

\renewcommand{\thesection}{\arabic{section}}

\newtheorem{theorem}{Theorem}
\newtheorem{lemma}{Lemme}

\renewcommand{\thetheorem}{\empty{}}
\renewcommand{\thelemma}{\empty{}} 

\newenvironment{proof}[1][Proof]{\begin{trivlist}
\item[\hskip \labelsep {\bfseries #1}]}{\end{trivlist}}
\newenvironment{definition}[1][Definition]{\begin{trivlist}
\item[\hskip \labelsep {\bfseries #1}]}{\end{trivlist}}
\newenvironment{example}[1][Example]{\begin{trivlist}
\item[\hskip \labelsep {\bfseries #1}]}{\end{trivlist}}
\newenvironment{remark}[1][Rq:]{\begin{trivlist}
\item[\hskip \labelsep {\bfseries #1}]}{\end{trivlist}}
\newenvironment{rappel}[1][rappel:]{\begin{trivlist}
\item[\hskip \labelsep {\bfseries #1}]}{\end{trivlist}}


\chapter*{Introduction}
\addcontentsline{toc}{chapter}{Introduction} 
Les différents concepts abordés sur la VC, ou Vapnik–Chervonenkis théorie: (fin 1970)
\begin{itemize}
\item Dimension VC
\item Pulvérisation
\item Échantillon Fantôme
\item Lemme de Sauer
\end{itemize}
On veut, avec une probabilité de $1 - \delta$:\\
$ \forall f \in F $ R(f)$ \leq $\^R (f)$ + \epsilon(\delta, F, n) $\\
\begin{tabbing}
avec : \= R(f) = $\mathds{P}_{X,Y \sim D}$(f(x)$\neq$y)\\
\> \^R(f) = $\frac{1}{n}\displaystyle { \sum_{i=1}^{n}}( \mathds{1}_{f(x_i)\neq y_i} $)\\
\end{tabbing}

\chapter*{Dimension VC}
\addcontentsline{toc}{chapter}{Dimension VC}

\section{Rappels et quelques exemples}
\subsection*{Rappels}
Lorsque |F| < $ +\infty $: avec proba 1-$ \delta $\\
$ \forall $ f $ \in $ F R(f)$ \leq $\^R(f) + $\sqrt{\frac{2}{n}*\log\frac{|F|}{\delta}}$\\
\newline
\underline{Preuve}:\\
\begin{itemize}
\item[$\bullet$] pour f fixé: $\mathds{P}($R(f)-\^R(f)$\geq \epsilon)\leq \exp ^{-2n\epsilon ^2}$
\item[$\bullet$] Borne de l'union: $\mathds{P}(\exists f\in F$ R(f)-\^R(f)$\geq \epsilon)\leq $ |F|$\exp ^{-2n\epsilon ^2}$\\
\end{itemize}

\begin{tiny}
Rappel du problème:\\
S = \{($X_i$, $Y_i)\}_{i=1}^n$ IID $ \sim $ D\\
\^R(f) = $\frac{1}{n}\displaystyle { \sum_{i=1}^{n}}( \mathds{1}_{f(x_i)\neq y_i} $)\\
R(f) = $\mathds{E}(\mathds{1}_{ f(x)\neq y })$ = $\mathds{P}$(f(x)$\neq$y)\\
\end{tiny}


\subsection*{Exemples sur deux classes de fonctions}
\begin{itemize}
\item[$\bullet$] $F_1 = \{f:\ f(x)=\mathds{1}_{x\geq t},\ t\in \mathds{R}\}
\ \cup \ \{f:\ f(x)=\mathds{1}_{x\leq t},\ t\in \mathds{R}\}$\\
\end{itemize}
$\rightarrow$ De combien de façons différentes peut-on étiqueter $x_1...x_n \in \mathds{R}$ grâce à $F_1$ ?\\
--> 2n ($\lll 2^n$)\\

\begin{itemize}
\item[$\bullet$] $F_2 = \{f:\ f(x)=\mathds{1}_{x\in A},\ A=[a,b]\times[c,d],\ a,b,c,d\ \in \ \mathds{R}\}$\\
\end{itemize}
$\rightarrow$ 2 points: toutes les 4 affectations sont réalisables (pulvérisable)\\
$\rightarrow$ 4 points (non alignés): toutes les 16 affectations sont réalisables (pulvérisable)\\
$\rightarrow$ 5 points: aucune configuration de 5 points n'est pulvérisée par $F_2$\\


\section{Théorème principal}
\begin{itemize}
\item[$\bullet$] F est une classe de fonctions $ \subseteq \mathcal{Y}^\mathcal{X}$\\
\item[$\bullet$] $\mathcal{N}_F (x_1..x_n)$ = $\{[f(x_1)..f(x_n)]\in \{0,i\}^n \ ,f\in F\}$\\
\end{itemize}

\begin{definition}
Coefficient de pulverisation $\mathcal{S}$(F, n)\\
$\mathcal{S}$(F,n) = $\underset{x_1..x_n}{max}|\mathcal{N}_F (x_1..x_n)|$\\
\end{definition}

\begin{remark}
$\mathcal{S}$(F,n) $ \leq 2^n$ mais bien souvent $\mathcal{S}$(F,n) $ \lll 2^n $\\
\end{remark}

\begin{definition}
Dimension de Vapnik–Chervonenkis\\
le plus grand k tq. $\mathcal{S}$(F,k) = $2^k$\\
\end{definition}

La VC dimension d'une classe de fonctions F, qu'on  notera VC(F), mesure la "richesse" de cette classe.\\
Si on revient sur nos exemples précédents:
\begin{itemize}
\item[]VC($F_1$) = 2
\item[]VC($F_1$) = 4\\
\end{itemize}

\begin{theorem}
VC (1979):\\
$\mathds{P}(sup|$\^R(f)-R(f)$| > \epsilon)\ \leq \ 8\mathcal{S}(F,n) \exp ^{-\frac{n\epsilon^2}{32}}$\\
$\mathds{P}(\exists f \in F:\ |$\^R(f)-R(f)$| > \epsilon)\ \leq \ 8\mathcal{S}(F,n)\exp ^{-\frac{n\epsilon^2}{32}}$\\
\end{theorem}

\begin{lemma}
Sauer-Shelah:\\
$\mathcal{S}$(F,n) $ \leq (n+1)^{VC(F)}$
\end{lemma}

\section{Preuve du théorème principal}
\begin{itemize}
\item[$\bullet$]On note \~f = $\underset{f\in F}{argmax}$|\^R(f)-R(f)|\\
\end{itemize}
\underline{Étape 1}:\\
$\mathds{P}(\underset{f\in F}{sup}|$\^R(f)-R(f)$| > \epsilon)\ \leq 2\mathds{P}_{SS'}(\underset{f\in F}{sup}|$\^R$_S(f)-$\^R$_{S'}(f)| > \frac{\epsilon}{2})$\\
\begin{tabbing}
$\mathds{P}(\underset{f\in F}{sup}|$\^R$_S(f)-$\^R$_{S'}(f)| \geq \frac{\epsilon}{2})$ \= $\geq \mathds{P}(|$\^R$_S($\~f$)-$\^R$_{S'}($\~f$)| \geq \frac{\epsilon}{2})$\\
\> $\geq \mathds{P}(|$\^R$_S($\~f$)-$R$($\~f$)| > \epsilon\ et\ |$\^R$_{S'}($\~f$)-$R$($\~f$)| < \frac{\epsilon}{2})$\\
\end{tabbing}
= $ \mathds{E}[\mathds{1}\{|$\^R$_S($\~f$)-$R$($\~f$)|>\epsilon\}\mathds{1}\{|$\^R$_{S'}($\~f$)-$R$($\~f$)|>\frac{\epsilon}{2}\}] $\\
= $ \mathds{E}_S \mathds{E}_{S|S'}[...] $\\
= $ \mathds{E}_S[\mathds{1}\{|$\^R$_S($\~f$)-$R$($\~f$)|\geq \epsilon\}\times \mathds{E}_{S|S'}\mathds{1}\{|$\^R$_{S'}($\~f$)-$R$($\~f$)|\leq \frac{\epsilon}{2}\}] $\\
= $ \mathds{E}_S[\mathds{1}\{|$\^R$_S($\~f$)-$R$($\~f$)|\geq \epsilon\}\ \mathds{P}_{S'|S}(|$\^R$_{S'}($\~f$)-$R$($\~f$)|\leq \frac{\epsilon}{2})] $\\

\begin{itemize}
\item[$\bullet$]supposons que S soit fixé:\\
\end{itemize}
\begin{tabbing}
\^R$_{S'}$(\~f)-R(\~f) \= = $\frac{1}{n}\displaystyle { \sum_{i=1}^{n}}[\mathds{1}\{$\~f$(x_i)\neq y_i'\}-\underset{X_i', Y_i'|S}{\mathds{E}}\mathds{1}\{$\~f$(X_i')\neq Y_i'\}]$\\
\> = $\frac{1}{n}\displaystyle{\sum_{i=1}^{n}}$ $\bigsqcup _{i|S}$\\
\> avec $\bigsqcup $, variable aléatoire IID, $\mathds{E}\bigsqcup _i = 0$\\
\end{tabbing}
\begin{rappel}
inégalité de Chebychev\\
Soit une v.a Z d’espérance $\mu $ et de variance $\sigma^2$\\
$\mathds{P}(|Z-\mu|\geq \alpha)\leq \frac{\sigma^2}{\alpha^2} $\\
\end{rappel}

\begin{tabbing}
$\mathds{P}_{S'|S}$\=$(|$\^R$_{S'}($\~f$)-$R$($\~f$)|< \frac{\varepsilon}{2})$\\
\> = $\mathds{P}_{S'|S}(|\frac{1}{n}\displaystyle{\sum_{i=1}^{n}}$ $\bigsqcup _{i}|< \frac{\varepsilon}{2})$\\
\> = $\mathds{P}_{S'|S}(|\sum \bigsqcup _{i}|< \frac{n \varepsilon}{2})$\\
\> = $\geq 1 - \frac{4}{n^2 \varepsilon ^2}\underset{S|S'}{var}(|\displaystyle{\sum_{i=1}^{n}}$ $ \bigsqcup _{i}|)$\\
\> = $\geq 1 - \frac{4}{n^2 \varepsilon ^2}\ n\times \underset{X_i', Y_i'}{var}(| \bigsqcup _{i}|)$\\
\> = $\geq 1 - \frac{4}{n^2 \varepsilon ^2}\ n\times \frac{1}{4}$ = $ 1 - \frac{4}{n \varepsilon ^2} \geq \frac{1}{2}$\\
\end{tabbing}
Si on fait l'hypothèse que $n\varepsilon ^2 \geq 2$\\
NB: $\frac{1}{4}$ est la variance maximale d'une variable de Bernoulli.\\
\newline
\underline{On a donc}:\\
$\mathds{P}(\underset{f\in F}{sup}|$\^R$_S(f)-$\^R$_{S'}(f)|>\frac{\varepsilon}{2})\geq \mathds{E}[\mathds{1}\{|$\^R$_S($\~f$)-$R$($\~f$)|>\varepsilon \}\ \mathds{P}_{S'|S}(|$\^R$_{S'}($\~f$)-$R$($\~f$)|>\frac{\varepsilon}{2})]$\\
$\geq \frac{1}{2} \mathds{E}[\mathds{1}\{|$\^R$_{S}($\~f$)-$R$($\~f$)|>\varepsilon\}]$\\
= $\frac{1}{2}\mathds{P}(|$\^R$_{S}($\~f$)-$R$($\~f$)|>\varepsilon) $\\
\newline
$\geq \frac{1}{2} \mathds{P}_S(\underset{f\in F}{sup}|$\^R$_{S}($f$)-$R$($f$)|>\varepsilon) $\\
\newline
et donc:\\
$ \mathds{P}_S(\underset{f\in F}{sup}|$\^R$_{S}($f$)-$R$($f$)|>\varepsilon)\ \leq \ 2\mathds{P}_{SS'}(\underset{f\in F}{sup}|$\^R$_{S}($f$)-$\^R$_{S'}($f$)|\geq \frac{\varepsilon}{2}) $\\
\begin{remark}
Dans cette dernière expression, S' est appelé un échantillon fantôme.\\
\end{remark}

\underline{Étape 2: Symétrisation}:\\
$ \mathds{P}_{SS'}(\underset{f\in F}{sup}|\frac{1}{n}\displaystyle {\sum_{i=1}^n}\mathds{1}\{f(x_i)\neq y_i\}-\frac{1}{n}\displaystyle {\sum_{i=1}^n}\mathds{1}\{f(X_i)\neq Y_i\}|\geq \frac{\varepsilon}{2}) $\\
= $ \mathds{P}_{SS'}(\underset{f\in F}{sup}\frac{1}{n}|\displaystyle {\sum_{i=1}^n}(\mathds{1}\{f(x_i)\neq y_i\}-\mathds{1}\{f(X_i)\neq Y_i\})|\geq \frac{\varepsilon}{2}) $\\
= $ \mathds{P}_{SS'\underline{\sigma}}(\underset{f\in F}{sup}\frac{1}{n}|\displaystyle {\sum_{i=1}^n}\sigma_i(\mathds{1}\{f(x_i)\neq y_i\}-\mathds{1}\{f(X_i)\neq Y_i\})|\geq \frac{\varepsilon}{2}) $\\
\newline
avec $\underline{\sigma}$ séquences $\sigma_1...\sigma_n$ de variables de Rademacher indépendantes:\\
$\mathds{P}(\sigma_i=+1)$ = $\mathds{P}(\sigma_i=-1)$ = $\frac{1}{2}$\\
\newline
$\leq \mathds{P}_{SS'\underline{\sigma}}(\underset{f\in F}{sup}\frac{1}{n}|\displaystyle {\sum_{i=1}^n}\sigma_i \mathds{1}\{f(x_i)\neq y_i\}|>\frac{\varepsilon}{4}\ ou\ \underset{f\in F}{sup}\frac{1}{n}|\displaystyle {\sum_{i=1}^n}\sigma_i \mathds{1}\{f(X_i)\neq Y_i\}|>\frac{\varepsilon}{4}) $\\
$\leq 2\mathds{P}_{S\sigma}(\underset{f\in F}{sup}\frac{1}{n}|\displaystyle {\sum_{i=1}^n}\sigma_i \mathds{1}\{f(X_i)\neq Y_i\}|\geq \frac{\varepsilon}{4}) $\\
\newline
On en est donc à:\\
\fbox{
\begin{minipage}{1\textwidth}
   $\mathds{P}(\underset{f\in F}{sup}|$\^R$_{S}($f$)-$R$($f$)|>\varepsilon) \leq 4\mathds{P}_{S\sigma}(\underset{f\in F}{sup}\frac{1}{n}|\displaystyle {\sum_{i=1}^n}\sigma_i \mathds{1}\{f(X_i)\neq Y_i\}|\geq \frac{\varepsilon}{4})$
\end{minipage}
}
\newline
\underline{Étape 3}: conclusion\\
\begin{itemize}
\item[$\bullet$]On s'intéresse à: $\frac{1}{n}|\displaystyle {\sum_{i=1}^n}\sigma_i \mathds{1}\{f(X_i)\neq Y_i\}|$\\
\item[$\bullet$]On suppose que $x_1...x_n$ et $y_1...y_n$ sont fixés.
\end{itemize}
Le vecteur (f($x_1$)...f($x_n$)) peut prendre au maximum $\mathcal{S}$(F,n) valeurs différentes.\\
\newline
Et donc:\\
($\mathds{1}\{$f($x_1$)$\neq y_1\}$...$\mathds{1}\{$f($x_n$)$\neq y_n\}$) peut prendre au maximum $\mathcal{S}$(F,n) valeurs différentes.\\
\newline
\begin{itemize}
\item[$\bullet$]$F_{x_1..x_n}\ \subseteq \ F$ le plus petit sous ensemble de $F$ tel que:\\
$\mathcal{N}_{F_{x_1..x_n}}(x_1..x_n) = \mathcal{N}_F(x_1..x_n)$\\
\end{itemize}
$\mid F_{x_1..x_n} \mid \ \leq \ \mathcal{S}$(F,n)\\
\newline
$ \mathds{P}(\underset{f\in F}{sup}\frac{1}{n}|\displaystyle {\sum_{i=1}^n}\sigma_i \mathds{1}\{f(x_i)\neq y_i\}|\geq \frac{\varepsilon}{4})$
= $\mathds{P}(\underset{f\in F_{x_1..x_n}}{max}\frac{1}{n}|\displaystyle {\sum_{i=1}^n}\sigma_i \mathds{1}\{f(x_i)\neq y_i\}|\geq \frac{\varepsilon}{4})$\\
= $\mathds{P}(\underset{f\in F_{x_1..x_n}}{\cup}\{\frac{1}{n}|\displaystyle {\sum_{i=1}^n}\sigma_i \mathds{1}\{f(x_i)\neq y_i\}|\geq \frac{\varepsilon}{4}\})$\\
$\leq \displaystyle {\sum_{f\in F_{x_1..x_n}}} \mathds{P}(\frac{1}{n}|\displaystyle {\sum_{i=1}^n}\sigma_i \mathds{1}\{f(x_i)\neq y_i\}|> \frac{\varepsilon}{4})$ (borne de l'union)\\
$\leq \mid F_{x_1..x_n} \mid \underset{f\in F_{x_1..x_n}}{sup} \mathds{P}(\frac{1}{n}|\displaystyle {\sum_{i=1}^n}\sigma_i \mathds{1}\{f(x_i)\neq y_i\}|> \frac{\varepsilon}{4}) $\\

$\leq \mathcal{S}(F,n) \underset{f\in F}{sup} \mathds{P}(\frac{1}{n}|\displaystyle {\sum_{i=1}^n}\sigma_i \mathds{1}\{f(x_i)\neq y_i\}|> \frac{\varepsilon}{4}) $\\
\newline
\underline{Étape 4}: Hoeffding ($x_i$ et $y_i$ fixés)\\
$\frac{1}{n}|\displaystyle {\sum_{i=1}^n}\underbrace{\sigma_i \mathds{1}\{f(X_i)\neq Y_i\}}_{A_i}|$\\
$A_i$ variable aléatoire (dépend de $\sigma_i$)\\
$A_i \in [-1, 1]$\\
$\mathds{E}A_i=0$\\
$A_i = \sigma_i K_i$\\
$\mathds{E}A_i = +1K_i \times \mathds{P}(\sigma_i=+1)-1K_i \times \mathds{P}(\sigma_i=-1)$ = $K_i \times \frac{1}{2}-K_i \times \frac{1}{2} = 0 $\\
\newline
$\mathds{P}(|\displaystyle {\sum_{i=1}^n}\sigma_i \mathds{1}\{f(x_i)\neq y_i\}|\geq \frac{n\varepsilon}{4})\leq 2exp(-\frac{2(\frac{n\varepsilon}{4})^2}{\displaystyle{\sum_{i=1}^n}(max\ A_i-min\ A_i)^2})$\\
$\leq 2exp(-\frac{2(\frac{n\varepsilon}{4})^2}{n\times4})= 2exp(-\frac{(\frac{n\varepsilon}{4})^2}{32})$\\

\end{document}