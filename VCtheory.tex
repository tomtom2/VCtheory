\documentclass[a4paper]{report}

\usepackage[utf8]{inputenc}
\usepackage[francais]{babel}
\usepackage[T1]{fontenc}
\usepackage{amsmath}
\usepackage{amsfonts}
\usepackage{amssymb}
\usepackage{placeins}
\usepackage{listings}
\usepackage{color}
\usepackage{textcomp}

%----- Package français
\usepackage[utf8]{inputenc} %reconnaissance des accents
\usepackage[francais]{babel} %document en français
\usepackage[T1]{fontenc} %codage des fonts TeX ?



%----- math
\usepackage{amsmath}

%----- images
\usepackage{graphicx}

\title{VC theory}
%\author{Thomas BRIEN}
%\date{28 janvier 2013}

\begin{document}
\maketitle
\tableofcontents



\chapter*{Introduction}
\addcontentsline{toc}{chapter}{Introduction} 
Les différents concepts abordés sur la VC, ou Vapnik–Chervonenkis théorie:
\begin{itemize}
\item Dimension VC
\item Pulvérisation
\item Échantillon Fantôme
\item Lemme de Sauer
\end{itemize}
On veut, avec une probabilité de $1 - \delta$:\\
$ \forall f \in F $ R(f)$ \leq $\^R (f)$ + \epsilon(\delta, F, n) $\\

\chapter*{chapitre}
\addcontentsline{toc}{chapter}{chapitre}
\section*{section}
\addcontentsline{toc}{section}{section}
\begin{tabbing}
1990' $ \Rightarrow$ \=$Hypertext Markup Language$ (HTML)\\
       \>$Hypertext Transfer Protocol$ (HTTP)\\
\end{tabbing}

\section*{Composition d'un site\\}
\addcontentsline{toc}{section}{Composition d'un site}
\begin{itemize}
\item les liens entre deux documents du site
\item les liens internes à un document
\item les liens vers l extérieur
\end{itemize}

\section*{Un exemple\\}
\addcontentsline{toc}{section}{Exemple}
\begin{verbatim}
<html><body>
	<lu>
		<li><a href="http://google.fr">lien vers Google</a></li>
		<li><a href="b">aller à la page b</a></li>
		<li><a href="#monAncre">aller à l'ancre</a></li>
	</lu>
	<div id="monAncre"></div>
</body></html>
\end{verbatim}

\paragraph*{un paragraph}
p


\chapter*{Bibliographie}
\addcontentsline{toc}{chapter}{Bibliographie}

\end{document}